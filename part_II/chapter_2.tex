\begin{fullwidth}
\chapter{\label{ch:super_methods}
The syllables superposition experiment}
\end{fullwidth}

\begin{chabstract}

In this chapter we present the experimental design and the data preprocessing and analysis methods employed.

\end{chabstract}

% notes
% introduction
% review the problem of binding from a connectionist perspective
% review the mechanism of the tensor framework and its operational implications. In particular the substraction of components.

\section{Experimental design}
%
% experimental design
% remember to mention in other sections the risks of the task itself.
% that we are looking for automatic responses and not forcing any
% particular operation that forces unification. Equally emphasizing both
% syllables too in the task to at least force complete processing.

\section{Data preprocessing}
% simple acquisition, preprocessing, glm processing methods

\section{Data analysis}
% Introduction to the searchlight methodology
% Derivation of language related and syllable discriminant network
% Searchlight classification models considered (demonstration of the superposition model with quick equations)
% Group level methodology presentation. how we dilate and compare models. how we extract clusters and how we populate their classification tables.

% Note for methods
%There are several sanity checks related to activation maps and classification of the \emph{Syllable task} that we could perform in sensory-motor regions.
% worrying in particular that the lack of a nested cross-validation fit of the hyper-parameters due to the severe time computational costs of the searchlight procedure.

\section{Coordinate based fMRI metanalysis of morphological effects}
% Introduce here how the coordinates were extracted from papers?
% Facilitates link in discussion? were clusters are linked to effects by their coordinates.


