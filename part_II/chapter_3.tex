\begin{fullwidth}
\chapter{\label{ch:super_syllables}
Experimental results}
\end{fullwidth}

\begin{chabstract}

In this chapter we report subjects behavioral performance ...

\end{chabstract}


\section{Behavioral performance}
%
The subjects had a behavioral performance above 97\% in both visual and auditory \emph{Pseudoword matching tasks}, except for \emph{Subject 05} that reported concentration span issues over all the acquisition.
Note that due to the experimental design structure, in which we only query few random samples, small score decrements can imply distraction over an important task segment.
\emph{Subjects 01 and 04} reported in the second auditory session that the volume was not high enough to be comfortable, although this did not reflect on their behavioral performance.
So we consider all subjects data apt to neuroimaging interpretation, with caution over \emph{Subject 05}. Behavioral performance details are provided in Table \ref{table:behavior}.


\begin{table}
\begin{tabular}{|>{\bfseries}l|rrr|}
\toprule
Subject &  Visual (\%) &  Auditory (\%) &  Overall (\%) \\
\midrule
01 &        97.22 &          97.22 &    97.22 \\
02 &       100.00 &          98.61 &    99.31 \\
03 &        97.92 &          97.22 &    97.57 \\
04 &        99.31 &          99.31 &    99.31 \\
05 &        92.36 &          88.89 &    90.62 \\
\bottomrule
\end{tabular}
\caption{\textbf{Behavioral performance on the \emph{Pseudowords matching task}:} Performance correspond to correctly identifying if the pseudowords were the same or different, with no answer considered as incorrect. Visual and Auditory headers refer to the sensory modality of the task, where overall is the mean performance of both modalities.}
\label{table:behavior}
\end{table}


\section{Sanity checks}\label{sec:sanity_checks}

\begin{figure*}[ht]
\scriptsize
\hspace{-4ex}
\begin{tabular}{ccc}
\textbf{\Large Subject 1} & \textbf{\Large Subject 2} & \textbf{\Large Subject 3}\\
{\includegraphics[width=.33\linewidth]{figures/part_II/langloc_01.pdf}}
\hspace{-1ex}
&{\includegraphics[width=.33\linewidth]{figures/part_II/langloc_03.pdf}}
\hspace{-1ex}
&{\includegraphics[width=.33\linewidth]{figures/part_II/langloc_04.pdf}}
\hspace{-1ex}\\
\rule{0pt}{6ex}
\textbf{\Large Subject 4} & \textbf{\Large Subject 5} & {}\\
{\includegraphics[width=.33\linewidth]{figures/part_II/langloc_05.pdf}}
\hspace{-1ex}
&{\includegraphics[width=.33\linewidth]{figures/part_II/langloc_06.pdf}}
\hspace{-1ex}
&{\includegraphics[width=.2\linewidth]{figures/part_II/langloc_legend.pdf}}
\hspace{-1ex} \\
\end{tabular}
\vspace{0ex}
\caption{\textbf{Language localizers:} We show left and right hemispheric contours of the language localizer contrast of word sequences over control stimuli (consonant strings or scrambled recordings), thresholded at a p-value < 10e-3.
Statistical images are projected in the anatomical space of each subject.}
\label{fig:language_localizers}
\end{figure*}

\paragraph{Language localizer activations:}
The contours of the language localizers' contrasts, thresholded at p-value < 10e-3, for both auditory and visual modalities are presented in Figure \ref{fig:language_localizers} for all subjects.
We also show in Figure \ref{fig:language_localizers_radial} the coverage of Mahowald et al. parcels\citep{mahowald2016reliable} by the thresholded language localizers for all subjects.
We see observe an expected left lateralization of the detected language network with more than 40\% coverage of all the language parcels, which covers the fronto-temporal language system that has been well depicted in previous imaging studies\citep{mahowald2016reliable, fedorenko2010new, dehaene2010learning, binder1997human}.
There is variability between the modalities, that particularly disfavors activations of the visual one, in which the subjects can get distracted from perceiving and processing the stimuli more easily, than in the auditory case.
This could be expected from the intrinsic variability of different experimental designs in language localizers as demonstrated by Mahowald et al.\citep{mahowald2016reliable}.
\emph{Subjects 1 and 5} have a defficient coverage that will diminish our capacity to interpret syllabic representation effects along their cortex.
In particular \emph{Subject 5}, who reported concentration problems, have an extremely defficient coverage of the language network.

\begin{figure}[ht]
\scriptsize
\hspace{-4ex}
\begin{tabular}{ccc}
{\includegraphics[width=0.1\linewidth]{figures/part_II/subjects_legend.pdf}}
\hspace{-1ex}
&{\includegraphics[width=0.45\linewidth]{figures/part_II/visual_langloc_radial.png}}
\hspace{-1ex}
&{\includegraphics[width=0.45\linewidth]{figures/part_II/auditory_langloc_radial.png}}
\hspace{-1ex}\\
\end{tabular}
\caption{\textbf{Language localizer parcel coverage:}
We show the parcel coverage of each language localizer for the 6 language parcels derived by Mahowald et al. in both hemispheres.
Each subject is represented in a radial chart to emphasize the overall coverage of the language localizers of each subject.
Also the left and right hemisphere parcels have been arranged symmetrically in the radial charts.}
\label{fig:language_localizers_radial}
\end{figure}

\paragraph{Motor activations:}
We verified the integrity of the activation maps of the \emph{Pseudoword matching task} with statistical tests portraying the left and right hand button press contrast.
Z score maps of the left over right button press contrast, for all subjects, are shown in Figure \ref{fig:button_press}, confirming a good statistical separation of hand responses.

\begin{figure}[ht]
\scriptsize
\hspace{-4ex}
\begin{tabular}{cccccl}
\textbf{\Large Subject 1} & \textbf{\Large Subject 2} & \textbf{\Large Subject 3} & \textbf{\Large Subject 4} & \textbf{\Large Subject 5} & {}\\
{\includegraphics[width=.14\linewidth]{figures/part_II/press_vis_01.pdf}}
\hspace{1ex}
&{\includegraphics[width=.14\linewidth]{figures/part_II/press_vis_03.pdf}}
\hspace{1ex}
&{\includegraphics[width=.14\linewidth]{figures/part_II/press_vis_04.pdf}}
\hspace{1ex}
&{\includegraphics[width=.14\linewidth]{figures/part_II/press_vis_05.pdf}}
\hspace{1ex}
&{\includegraphics[width=.14\linewidth]{figures/part_II/press_vis_06.pdf}}
\hspace{1ex}
& {} \\
{\includegraphics[width=.14\linewidth]{figures/part_II/press_aud_01.pdf}}
\hspace{1ex}
&{\includegraphics[width=.14\linewidth]{figures/part_II/press_aud_03.pdf}}
\hspace{1ex}
&{\includegraphics[width=.14\linewidth]{figures/part_II/press_aud_04.pdf}}
\hspace{1ex}
&{\includegraphics[width=.14\linewidth]{figures/part_II/press_aud_05.pdf}}
\hspace{1ex}
&{\includegraphics[width=.14\linewidth]{figures/part_II/press_aud_06.pdf}}
\hspace{1ex}
&{\includegraphics[width=.10\linewidth]{figures/part_II/press_legend.pdf}} \\
\end{tabular}
%\vspace{3ex}
\caption{\textbf{Button press effects:} We show the left button press over right button press contrast Z scores from the auditory modality, thresholded at p < 10e-4, for all subjects. Statistical images correspond to the anatomical space of each subject.}
\label{fig:button_press}
\end{figure}

We also verified that we can employ a Support Vector Classifier (SVC) to distinguish individual left and right button press activation maps derived from the \emph{Pseudoword matching task} General Linear Model (GLM).
As can be seen in Table \ref{table:clic}, we achieve high classification scores of right and left button press events for all subjects.
Moreover as would be expected, the classification generalize across sensory modalities.

\begin{table}
\begin{tabular}{|>{\bfseries}l|rrrrr|}
\toprule
(Train, Test) & (V, V) & (A, A) & (V, A) & (A, V) & (V-A, V-A) \\
Subject & (\%) &  (\%)   & (\%)  & (\%)  &   (\%)\\
\midrule
01      &   84.38*** &   93.50*** &   80.84*** &   76.94*** &           90.88*** \\
02      &   95.38*** &   92.50*** &   84.03*** &   93.03*** &           95.31*** \\
03      &   98.00*** &   99.00*** &   93.91*** &   98.75*** &          100.00*** \\
04      &   97.38*** &   99.50*** &   97.50*** &   98.72*** &          100.00*** \\
05      &   86.62*** &   77.62*** &   90.12*** &   74.28*** &           93.56*** \\
\bottomrule
\end{tabular}
\vspace{5ex}
\caption{\textbf{Classification of left and right button press maps of \emph{Pseudoword matching task}:} \emph{"V"} corresponds to the Visual modality and \emph{"A"} to the Auditory modality. \emph{"V-A"} corresponds to pooling together both datasets for training and testing.}
\label{table:clic}
\end{table}

\blankfootnote{\emph{chance: 50\% \\ * : p < 10e-2,\\ ** : p < 10e-3,\\ *** : p < 10e-4 \\ Bonferroni corrected for 25 similar tests performed}}

\paragraph{Visual activations:}
We verified the statistical effects of left and right syllable position in the Visual hOc1 region.
In the experimental design we asked the subjects to fixate a centered green dot before stimuli presentation.
So we expected, from well known retinotopic effects in the primary visual cortex \citep{tootell1998retinotopy}, to see an hemispheric partition of left and right syllable position effects, such that left syllable effects would be emphasized in the right hemisphere and right syllable effects in the left hemisphere.
This was the case, as can be seen in Figure \ref{fig:retinotopy}, where \emph{Subjects 1 and 4} have the clearest retinotopic activations.
Nonetheless we can also appreciate in the images that some subjects did not manage to completely follow the fixation instruction, since effects of both positions are present together in both hemispheres.


\begin{figure*}[ht]
\scriptsize
\vspace{2ex}
\hspace{-4ex}
\begin{tabular}{cccccl}
\textbf{\Large Subject 1} & \textbf{\Large Subject 2} & \textbf{\Large Subject 3} & \textbf{\Large Subject 4} & \textbf{\Large Subject 5} & {}\\
{\includegraphics[width=.13\linewidth]{figures/part_II/retinotopy_01.pdf}}
\hspace{1ex}
&{\includegraphics[width=.13\linewidth]{figures/part_II/retinotopy_03.pdf}}
\hspace{1ex}
&{\includegraphics[width=.13\linewidth]{figures/part_II/retinotopy_04.pdf}}
\hspace{1ex}
&{\includegraphics[width=.13\linewidth]{figures/part_II/retinotopy_05.pdf}}
\hspace{1ex}
&{\includegraphics[width=.13\linewidth]{figures/part_II/retinotopy_06.pdf}}
\hspace{-1ex}
&{\includegraphics[width=.12\linewidth]{figures/part_II/retinotopy_legend.pdf}}
\hspace{-1ex} \\
\end{tabular}
\vspace{3ex}
\caption{\textbf{Retinotopic effect:} We show first and second syllable position effects masked by the Visual hOc1 region, thresholded at a p-value < 0.005.
Statistical images correspond to the anatomical space of each subject.}
\label{fig:retinotopy}
\end{figure*}


\section{Visual regions}

\subsection{Visual hOc1 region (Primary visual area V1)}

The size of the text presented in the \emph{Pseudowords matching task} allowed us to induce enough sparsity in the voxel activations due to retinotopic mapping.
Thanks to this we achieved significant classification scores for almost all condition categories in all subjects, with subject 4 having an exemplary performance, distinguishing significantly all conditions in all classifiers.
We show in Figure \ref{fig:visual_classification} accuracy scores from which chance baseline was substracted for each condition.
All the classifiers were trained on the visual stimuli and were not able to generalize to auditory stimuli, as would be expected from primary visual areas.
Significant scores are marked with a star in case of a p-value < 0.05.
We observe, from the relative area of accuracy above chance, that we could decode syllables in each position and pseudowords best in Subjects 1 and 4.
Moreover Subject 5, that reported problems with attention, had the worst classifier performance.


\hspace{-5ex}
\begin{figure*}[ht]
\scriptsize
\hspace{0ex}
\begin{tabular}{lccc}
{} & \textbf{\Large Pseudowords} & \textbf{\Large Syllables in position (F or S)} & \textbf{\Large Confusion cell categories}\\
\vspace{2ex}
\hspace{0ex}
{\includegraphics[width=.07\linewidth]{figures/part_II/subjects_legend.pdf}}
\hspace{-3ex}
&{\includegraphics[width=.3\linewidth]{figures/part_II/roi_plots/visual/radial_word.pdf}}
\hspace{-3ex}
&{\includegraphics[width=.3\linewidth]{figures/part_II/roi_plots/visual/radial_joint.pdf}}
\hspace{-3ex}
&{\includegraphics[width=.3\linewidth]{figures/part_II/roi_plots/visual/super_tests.pdf}}
\hspace{0ex}
\\
\end{tabular}
% \vspace{2ex}
\fullcaption{\textbf{Confusion matrix analysis in visual regions:}
Chance baseline has been substracted from all accuracy scores. Chance is 11.11\% for pseudoword conditions and 33.33\% for positioned syllable conditions. We show at the left the accuracy for each of the conditions in the pseudoword classifier. At the center we show accuracy for the classifiers of first (F) and second (S) position syllables. The accuracy score points are denoted with stars whenever they are significant. At the right we show the mean confusion value of three types of cells (Overlap categories) in the pseudoword classifier confusion matrix. First those that represent confusion between conditions that have overlapping syllables in a position. Second those between conditions that share syllables in different positions. Third those between conditions that share no syllables. The mean confusion values are denoted with stars whenever they are significant. Black stars on top of the lines connecting overlap category values represent significant differences between them. Significance represents a p-value < 0.05 derived from the shuffled models}
\label{fig:visual_classification}
\end{figure*}
% \vspace{-1ex}


We also observe in Figure \ref{fig:visual_classification}, patterns of superposed representations from the mean confusion of the different confusion matrix cell groups explained in Methods section \ref{sec:methods_structural}.
In general all subjects show a pattern with higher confusion values for conditions with position overlapping syllables.
Subjects 1 and 4 portray strong evidence for superposition, since they have a significant decrease of confusion between pseudoword categories that have a crossed or no syllable overlap, leading to a significant difference between the mean confusion values.


\begin{figure*}[ht]
\scriptsize
\vspace{2ex}
\hspace{-4ex}
\begin{tabular}{ccccc}
\textbf{\Large Subject 1} & \textbf{\Large Subject 2} & \textbf{\Large Subject 3} & \textbf{\Large Subject 4} & \textbf{\Large Subject 5}\\
{\includegraphics[width=.19\linewidth]{figures/part_II/locality/visual/locality_test_01.png}}
\hspace{0ex}
&{\includegraphics[width=.19\linewidth]{figures/part_II/locality/visual/locality_test_03.png}}
\hspace{0ex}
&{\includegraphics[width=.19\linewidth]{figures/part_II/locality/visual/locality_test_04.png}}
\hspace{0ex}
&{\includegraphics[width=.19\linewidth]{figures/part_II/locality/visual/locality_test_05.png}}
\hspace{0ex}
&{\includegraphics[width=.19\linewidth]{figures/part_II/locality/visual/locality_test_06.png}}
\hspace{-1ex} \\
\end{tabular}
\vspace{3ex}
\caption{\textbf{Partitioned position representations in visual regions:}
We show in black the overlap of the "n" best voxels selected by the two syllable position classifiers. In red we show the overlap distribution given by the shuffled models. In green we denote segments of significantly inferior overlap with a p-value < 0.05 with respect to the shuffled distribution.}
\label{fig:visual_locality}
\end{figure*}


As would be expected from the hemispheric projection of retinotopic representations, we observe in Figure \ref{fig:visual_locality} significant patterns of partitioned position representations in all Subjects except Subject 5.
This means that the best voxels overlap of the two syllable position classifiers, given by the coefficient weights of the classifers, was less than chance for several number of "n" best voxels.
The pattern is best represented by Subject 4 that had the most accurate models.


\subsection{The Visual Word Form Area (VWFA)}

A special case of visual regions that is also part of the language system is the Visual Word Form Area.
Contrary to other language regions in which we would expect abstract amodal representations, the VWFA receives its name for its capacity to recognize letters and digits during reading, so we analyzed classifiers in this region trained only on visual stimuli.
We observed that no Subject has significant accuracy scores across all conditions in any classifier, as can be seen in Figure \ref{fig:vwfa_classification}.
Some positioned syllable classifiers were only significantly accurate for one condition and the syllable "fi" had particularly high accuracy scores in the models of Subjects 1 and 5.
The classifiers were not able to generalize to auditory stimuli.

Patterns of the confusion cell overlap categories suggest evidence against superposition in this area.
We observe in the right plot of Figure \ref{fig:vwfa_classification} a constant pattern across all subjects, in which the mean confusion of pseudowords sharing syllables in different positions is higher than those with overlapping syllables in the same position or with no syllable overlap.
We did not find any significant patterns in favor or against partitioned position representations in any of the Subjects, which can not either be interpreted as evidence for distributed representations.


\hspace{-5ex}
\begin{figure*}[ht]
\scriptsize
\hspace{0ex}
\begin{tabular}{lccc}
{} & \textbf{\Large Pseudowords} & \textbf{\Large Syllables in position (F or S)} & \textbf{\Large Confusion cell categories}\\
\vspace{2ex}
\hspace{0ex}
{\includegraphics[width=.07\linewidth]{figures/part_II/subjects_legend.pdf}}
\hspace{-3ex}
&{\includegraphics[width=.3\linewidth]{figures/part_II/roi_plots/vwfa/radial_word.pdf}}
\hspace{-3ex}
&{\includegraphics[width=.3\linewidth]{figures/part_II/roi_plots/vwfa/radial_joint.pdf}}
\hspace{-3ex}
&{\includegraphics[width=.3\linewidth]{figures/part_II/roi_plots/vwfa/super_tests.pdf}}
\hspace{0ex}
\\
\end{tabular}
% \vspace{2ex}
\fullcaption{\textbf{Confusion matrix analysis in VWFA:}
Chance baseline has been substracted from all accuracy scores. Chance is 11.11\% for pseudoword conditions and 33.33\% for positioned syllable conditions. We show at the left the accuracy for each of the conditions in the pseudoword classifier. At the center we show accuracy for the classifiers of first (F) and second (S) position syllables. The accuracy score points are denoted with stars whenever they are significant. At the right we show the mean confusion value of three types of cells (Overlap categories) in the pseudoword classifier confusion matrix. First those that represent confusion between conditions that have overlapping syllables in a position. Second those between conditions that share syllables in different positions. Third those between conditions that share no syllables. The mean confusion values are denoted with stars whenever they are significant. Black stars on top of the lines connecting overlap category values represent significant differences between them. Significance represents a p-value < 0.05 derived from the shuffled models}
\label{fig:vwfa_classification}
\end{figure*}
% \vspace{-1ex}


\section{Auditory Te10 region}



\hspace{-5ex}
\begin{figure*}[ht]
\scriptsize
\hspace{0ex}
\begin{tabular}{lccc}
{} & \textbf{\Large Pseudowords} & \textbf{\Large Syllables in position (F or S)} & \textbf{\Large Confusion cell categories}\\
\vspace{2ex}
\hspace{0ex}
{\includegraphics[width=.07\linewidth]{figures/part_II/subjects_legend.pdf}}
\hspace{-3ex}
&{\includegraphics[width=.3\linewidth]{figures/part_II/roi_plots/auditory/radial_word.pdf}}
\hspace{-3ex}
&{\includegraphics[width=.3\linewidth]{figures/part_II/roi_plots/auditory/radial_joint.pdf}}
\hspace{-3ex}
&{\includegraphics[width=.3\linewidth]{figures/part_II/roi_plots/auditory/super_tests.pdf}}
\hspace{0ex}
\\
\end{tabular}
% \vspace{2ex}
\fullcaption{\textbf{Confusion matrix analysis in the Auditory Te10 region:}
Chance baseline has been substracted from all accuracy scores. Chance is 11.11\% for pseudoword conditions and 33.33\% for positioned syllable conditions. We show at the left the accuracy for each of the conditions in the pseudoword classifier. At the center we show accuracy for the classifiers of first (F) and second (S) position syllables. The accuracy score points are denoted with stars whenever they are significant. At the right we show the mean confusion value of three types of cells (Overlap categories) in the pseudoword classifier confusion matrix. First those that represent confusion between conditions that have overlapping syllables in a position. Second those between conditions that share syllables in different positions. Third those between conditions that share no syllables. The mean confusion values are denoted with stars whenever they are significant. Black stars on top of the lines connecting overlap category values represent significant differences between them. Significance represents a p-value < 0.05 derived from the shuffled models.}
\label{fig:vwfa_classification}
\end{figure*}
% \vspace{-1ex}


\section{Language regions}


\subsection{Language regions derived from constituency effects (Pallier et al.)}


\hspace{-5ex}
\begin{figure*}[ht]
\scriptsize
\hspace{0ex}
\begin{tabular}{lcccc}
{} & \textbf{\Large [V] Pseudowords} & \textbf{\Large [A] Pseudowords} & \textbf{\Large [V] Syllables (F or S)} & \textbf{\Large [A] Syllables (F or S)}\\
\hspace{-4ex}
{\includegraphics[width=.07\linewidth]{figures/part_II/subjects_legend.pdf}}
\hspace{-4ex}
&{\includegraphics[width=.23\linewidth]{figures/part_II/roi_plots/pSTS/Vis/radial_word.pdf}}
\hspace{-4ex}
&{\includegraphics[width=.23\linewidth]{figures/part_II/roi_plots/pSTS/Aud/radial_word.pdf}}
\hspace{-4ex}
&{\includegraphics[width=.23\linewidth]{figures/part_II/roi_plots/pSTS/Vis/radial_joint.pdf}}
\hspace{-4ex}
&{\includegraphics[width=.23\linewidth]{figures/part_II/roi_plots/pSTS/Aud/radial_joint.pdf}}
\hspace{-4ex}
\\
\hspace{-4ex}
{}
\hspace{-4ex}
&{\includegraphics[width=.23\linewidth]{figures/part_II/roi_plots/IFGorb/Vis/radial_word.pdf}}
\hspace{-4ex}
&{\includegraphics[width=.23\linewidth]{figures/part_II/roi_plots/IFGorb/Aud/radial_word.pdf}}
\hspace{-4ex}
&{\includegraphics[width=.23\linewidth]{figures/part_II/roi_plots/IFGorb/Vis/radial_joint.pdf}}
\hspace{-4ex}
&{\includegraphics[width=.23\linewidth]{figures/part_II/roi_plots/IFGorb/Aud/radial_joint.pdf}}
\hspace{-4ex}
\\
\hspace{-4ex}
{}
\hspace{-4ex}
&{\includegraphics[width=.23\linewidth]{figures/part_II/roi_plots/IFGtri/Vis/radial_word.pdf}}
\hspace{-4ex}
&{\includegraphics[width=.23\linewidth]{figures/part_II/roi_plots/IFGtri/Aud/radial_word.pdf}}
\hspace{-4ex}
&{\includegraphics[width=.23\linewidth]{figures/part_II/roi_plots/IFGtri/Vis/radial_joint.pdf}}
\hspace{-4ex}
&{\includegraphics[width=.23\linewidth]{figures/part_II/roi_plots/IFGtri/Aud/radial_joint.pdf}}
\hspace{-4ex}
\\
\end{tabular}
% \vspace{2ex}
\fullcaption{\textbf{Accuracy in syntactic constituency regions:}
Chance baseline has been substracted from all accuracy scores. Chance is 11.11\% for pseudoword conditions and 33.33\% for positioned syllable conditions. We show at the left the accuracy for each of the conditions in the pseudoword classifier. At the center we show accuracy for the classifiers of first (F) and second (S) position syllables. The accuracy score points are denoted with stars whenever they are significant.}
\label{fig:vwfa_classification}
\end{figure*}
% \vspace{-1ex}



\hspace{-5ex}
\begin{figure*}[ht]
\scriptsize
\hspace{0ex}
\begin{tabular}{lccc}
{} & \textbf{\Large Pseudowords} & \textbf{\Large Syllables in position (F or S)} & \textbf{\Large Confusion cell categories}\\
\vspace{2ex}
\hspace{0ex}
{\includegraphics[width=.07\linewidth]{figures/part_II/subjects_legend.pdf}}
\hspace{-3ex}
&{\includegraphics[width=.3\linewidth]{figures/part_II/roi_plots/auditory/radial_word.pdf}}
\hspace{-3ex}
&{\includegraphics[width=.3\linewidth]{figures/part_II/roi_plots/auditory/radial_joint.pdf}}
\hspace{-3ex}
&{\includegraphics[width=.3\linewidth]{figures/part_II/roi_plots/auditory/super_tests.pdf}}
\hspace{0ex}
\\
\end{tabular}
% \vspace{2ex}
\fullcaption{\textbf{Confusion matrix syllable overlap categories of syntactic constituency regions:}
We show the mean confusion value of three types of cells (Overlap categories) in the pseudoword classifier confusion matrix. First those that represent confusion between conditions that have overlapping syllables in a position. Second those between conditions that share syllables in different positions. Third those between conditions that share no syllables. The mean confusion values are denoted with stars whenever they are significant. Black stars on top of the lines connecting overlap category values represent significant differences between them. Significance represents a p-value < 0.05 derived from the shuffled models.}
\label{fig:vwfa_classification}
\end{figure*}
% \vspace{-1ex}





\hspace{-5ex}
\begin{figure*}[ht]
\scriptsize
\hspace{0ex}
\begin{tabular}{lcccc}
{} & \textbf{\Large [V] Pseudowords} & \textbf{\Large [A] Pseudowords} & \textbf{\Large [V] Syllables (F or S)} & \textbf{\Large [A] Syllables (F or S)}\\
\hspace{-4ex}
{\includegraphics[width=.07\linewidth]{figures/part_II/subjects_legend.pdf}}
\hspace{-4ex}
&{\includegraphics[width=.23\linewidth]{figures/part_II/roi_plots/aSTS/Vis/radial_word.pdf}}
\hspace{-4ex}
&{\includegraphics[width=.23\linewidth]{figures/part_II/roi_plots/aSTS/Aud/radial_word.pdf}}
\hspace{-4ex}
&{\includegraphics[width=.23\linewidth]{figures/part_II/roi_plots/aSTS/Vis/radial_joint.pdf}}
\hspace{-4ex}
&{\includegraphics[width=.23\linewidth]{figures/part_II/roi_plots/aSTS/Aud/radial_joint.pdf}}
\hspace{-4ex}
\\
\hspace{-4ex}
{}
\hspace{-4ex}
&{\includegraphics[width=.23\linewidth]{figures/part_II/roi_plots/TP/Vis/radial_word.pdf}}
\hspace{-4ex}
&{\includegraphics[width=.23\linewidth]{figures/part_II/roi_plots/TP/Aud/radial_word.pdf}}
\hspace{-4ex}
&{\includegraphics[width=.23\linewidth]{figures/part_II/roi_plots/TP/Vis/radial_joint.pdf}}
\hspace{-4ex}
&{\includegraphics[width=.23\linewidth]{figures/part_II/roi_plots/TP/Aud/radial_joint.pdf}}
\hspace{-4ex}
\\
\hspace{-4ex}
{}
\hspace{-4ex}
&{\includegraphics[width=.23\linewidth]{figures/part_II/roi_plots/TPJ/Vis/radial_word.pdf}}
\hspace{-4ex}
&{\includegraphics[width=.23\linewidth]{figures/part_II/roi_plots/TPJ/Aud/radial_word.pdf}}
\hspace{-4ex}
&{\includegraphics[width=.23\linewidth]{figures/part_II/roi_plots/TPJ/Vis/radial_joint.pdf}}
\hspace{-4ex}
&{\includegraphics[width=.23\linewidth]{figures/part_II/roi_plots/TPJ/Aud/radial_joint.pdf}}
\hspace{-4ex}
\\
\end{tabular}
% \vspace{2ex}
\fullcaption{\textbf{Accuracy in semantic constituency regions:}
Chance baseline has been substracted from all accuracy scores. Chance is 11.11\% for pseudoword conditions and 33.33\% for positioned syllable conditions. We show at the left the accuracy for each of the conditions in the pseudoword classifier. At the center we show accuracy for the classifiers of first (F) and second (S) position syllables. The accuracy score points are denoted with stars whenever they are significant.}
\label{fig:vwfa_classification}
\end{figure*}
% \vspace{-1ex}











\subsection{Broca 44 and 45}


\hspace{-5ex}
\begin{figure*}[ht]
\scriptsize
\hspace{0ex}
\begin{tabular}{lcccc}
{} & \textbf{\Large [V] Pseudowords} & \textbf{\Large [A] Pseudowords} & \textbf{\Large [V] Syllables (F or S)} & \textbf{\Large [A] Syllables (F or S)}\\
\vspace{2ex}
\hspace{-4ex}
{\includegraphics[width=.07\linewidth]{figures/part_II/subjects_legend.pdf}}
\hspace{-4ex}
&{\includegraphics[width=.23\linewidth]{figures/part_II/roi_plots/Broca44/Vis/radial_word.pdf}}
\hspace{-4ex}
&{\includegraphics[width=.23\linewidth]{figures/part_II/roi_plots/Broca44/Aud/radial_word.pdf}}
\hspace{-4ex}
&{\includegraphics[width=.23\linewidth]{figures/part_II/roi_plots/Broca44/Vis/radial_joint.pdf}}
\hspace{-4ex}
&{\includegraphics[width=.23\linewidth]{figures/part_II/roi_plots/Broca44/Aud/radial_joint.pdf}}
\hspace{-4ex}
\\
\hspace{-4ex}
{}
\hspace{-4ex}
&{\includegraphics[width=.23\linewidth]{figures/part_II/roi_plots/Broca45/Vis/radial_word.pdf}}
\hspace{-4ex}
&{\includegraphics[width=.23\linewidth]{figures/part_II/roi_plots/Broca45/Aud/radial_word.pdf}}
\hspace{-4ex}
&{\includegraphics[width=.23\linewidth]{figures/part_II/roi_plots/Broca45/Vis/radial_joint.pdf}}
\hspace{-4ex}
&{\includegraphics[width=.23\linewidth]{figures/part_II/roi_plots/Broca45/Aud/radial_joint.pdf}}
\hspace{-4ex}
\\
\end{tabular}
% \vspace{2ex}
\fullcaption{\textbf{Conditions accuracy in Broca 44 and 45:}
Chance baseline has been substracted from all accuracy scores. Chance is 11.11\% for pseudoword conditions and 33.33\% for positioned syllable conditions. We show at the left the accuracy for each of the conditions in the pseudoword classifier. At the center we show accuracy for the classifiers of first (F) and second (S) position syllables. The accuracy score points are denoted with stars whenever they are significant.}
\label{fig:vwfa_classification}
\end{figure*}
% \vspace{-1ex}




% \begin{figure}[ht]
% \scriptsize
% \vspace{2ex}
% \includegraphics[width=1.\linewidth]{figures/part_II/confusion_subject_4.png}
% \vspace{3ex}
% \caption{\textbf{Confusion matrices of \emph{Subject 4}:}
% We show }
% \label{fig:visual_confusion}
% \end{figure}



% \paragraph{\emph{Pseudoword matching task} effects:}
% Neural activations related to differences between the nine bi-syllabic conditions are spread across the cortex for all subjects, as reveals the simple check of the contour of an F test, thresholded at p-value < 0.0001, of any difference between conditions, shown in Figure \ref{fig:any-effects}.
% Nonetheless, to pursue the objective of this work, we needed to separate regions encoding the stimuli as a whole (holistic representation) from regions encoding stimuli parts (distributed representation) that might further follow the superposition principle (superposed representation).
% From an statistical point of view, this means that we wanted to discriminate regions on which there are only effects of the first and second syllable position from regions on which there is an interaction effect that would suggest additional encoding terms contradicting the superposition principle.
% In Figure \ref{fig:syllable_effects} we pool together the filled contours of the statistical effects, thresholded at p-value < 0.0001, of any syllable position and their interaction from both auditory and visual modalities.
% From the high level perspective of Figure \ref{fig:syllable_effects}, we can observe that all effects spread around the whole brain and then occupy similar functional regions in all subjects, although position effects seem to occupy a higher portion of the cortex.

% \begin{figure}[ht]
% \scriptsize
% \vspace{5ex}
% \hspace{-4ex}
% \begin{tabular}{ccccc}
% \textbf{\Large Subject 1} & \textbf{\Large Subject 2} & \textbf{\Large Subject 3} & \textbf{\Large Subject 4} & \textbf{\Large Subject 5}\\
% {\includegraphics[width=.165\linewidth]{figures/part_II/any-effects_01.pdf}}
% \hspace{1ex}
% &{\includegraphics[width=.165\linewidth]{figures/part_II/any-effects_03.pdf}}
% \hspace{1ex}
% &{\includegraphics[width=.165\linewidth]{figures/part_II/any-effects_04.pdf}}
% \hspace{1ex}
% &{\includegraphics[width=.165\linewidth]{figures/part_II/any-effects_05.pdf}}
% \hspace{1ex}
% &{\includegraphics[width=.165\linewidth]{figures/part_II/any-effects_06.pdf}}
% \hspace{1ex}\\
% \end{tabular}
% \vspace{3ex}
% \caption{\textbf{Any stimuli difference:} We show the contour of the statistical F test reflecting any difference between the bi-syllabic conditions, thresholded at a p-value < 0.0001. Statistical images correspond to the anatomical space of each subject.}
% \label{fig:any-effects}
% \end{figure}


% \begin{figure*}[ht]
% \scriptsize
% \hspace{-4ex}
% \begin{tabular}{ccc}
% \textbf{\Large Subject 1} & \textbf{\Large Subject 2} & \textbf{\Large Subject 3}\\
% {\includegraphics[width=.33\linewidth]{figures/part_II/all-effects_01.pdf}}
% \hspace{-1ex}
% &{\includegraphics[width=.33\linewidth]{figures/part_II/all-effects_03.pdf}}
% \hspace{-1ex}
% &{\includegraphics[width=.33\linewidth]{figures/part_II/all-effects_04.pdf}}
% \hspace{-1ex}\\
% \rule{0pt}{6ex}
% \textbf{\Large Subject 4} & \textbf{\Large Subject 5} & {}\\
% {\includegraphics[width=.33\linewidth]{figures/part_II/all-effects_05.pdf}}
% \hspace{-1ex}
% &{\includegraphics[width=.33\linewidth]{figures/part_II/all-effects_06.pdf}}
% \hspace{-1ex}
% &{\includegraphics[width=.2\linewidth]{figures/part_II/all-effects_legend.pdf}}
% \hspace{-1ex} \\
% \end{tabular}
% \vspace{3ex}
% \caption{\textbf{Pseudoword matching task effects:} We show blobs corresponding to the statistical test of main effects (effect of first and second position) and interaction (sign of holistic representation) for the \emph{Pseudoword matching task}, thresholded at a p-value < 0.0001.
% We consider blobs from both auditory and visual modalities of the task together, to portray the extent of the possible effects. Statistical images correspond to in the anatomical space of each subject.}
% \label{fig:syllable_effects}
% \end{figure*}


% \paragraph{Searchlight networks derived:}
% Since we designed our stimuli to be interpreted from the point of view of morphological processing (language processing), we employed the statistical effect maps from both modalities to determine a sub-network of the identified language network in each subject that would be used for further classification with the searchlight procedure presented in the methods section \ref{sec:searchlight_classification}.
% We show in in Figure \ref{fig:searchlight_regions} the obtained searchlight networks for each subject, while emphasizing that both holistic and superposed representation candidate regions are captured inside the identified language networks.
% Notice that both candidates are spread on the whole fronto-temporal language system and that \emph{Subjects 2 and 4} have broad searchlight networks corresponding to their broader language network activations, as depicted in Figure \ref{fig:language_localizers}.


% \begin{figure}[ht]
% \scriptsize
% \hspace{-4ex}
% \begin{tabular}{cc}
% \textbf{\Large Subject 1} & \textbf{\Large Subject 2}\\
% {\includegraphics[width=.5\linewidth]{figures/part_II/searchlight-regions_01.pdf}}
% \hspace{-1ex}
% &{\includegraphics[width=.5\linewidth]{figures/part_II/searchlight-regions_03.pdf}}
% \hspace{-1ex}\\
% \rule{0pt}{6ex}
% \textbf{\Large Subject 3} & \textbf{\Large Subject 4}\\
% {\includegraphics[width=.5\linewidth]{figures/part_II/searchlight-regions_04.pdf}}
% \hspace{-1ex}
% &{\includegraphics[width=.5\linewidth]{figures/part_II/searchlight-regions_05.pdf}}
% \hspace{-1ex}\\
% \rule{0pt}{6ex}
% \textbf{\Large Subject 5} & {}\\
% {\includegraphics[width=.5\linewidth]{figures/part_II/searchlight-regions_06.pdf}}
% \hspace{-1ex}
% &{\includegraphics[width=.3\linewidth]{figures/part_II/searchlight-regions_legend.pdf}}
% \hspace{-1ex} \\
% \end{tabular}
% \vspace{3ex}
% \caption{\textbf{Searchlight networks:} We show the derived searchlight network for each subject. We separate for illustrative purposes the portion of the network derived from interaction effects (Holistic candidate) and from exclusive main effects (Superposition candidate). Images correspond to the anatomical space of each subject.}
% \label{fig:searchlight_regions}
% \end{figure}
