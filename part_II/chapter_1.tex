\begin{fullwidth}
\chapter{\label{ch:super_intro}
The superposition principle and Neuroimaging}
\end{fullwidth}

\begin{chabstract}

In this chapter we introduce .

\end{chabstract}


\section{Smolensky's tensor framework and variable binding}


\section{The superposition principle from Smolensky's tensor framework}


\subsection{Neural interpretation}


\subsection{Bold-fMRI interpretation}
% review the problem of binding from a connectionist perspective with Smolensky tensors


\subsection{Neuroimaging of superposition}
% MAYBE MERGE WITH PREVIOUS POINT, it would be a comment on previous cases of superposition demonstrated with Bold-FMRI experiments.

% Superposition related experiments, not necessarily language related?
% The bag of words classification that works well in practice? Counterevidence of superposition
% trivial evidence in sensory systems like retinotopy?
% Quick introduction to the language network with emphasis of the Federonko language network localizers metanalysis.
% Since we are working with syllables, the morphology effects metaanalysis literature is relevant. Coordinates of previous effects. Would we expect superposition in some type of effects? (Careful not to overlap with discussion)

\section{The superposition principle applied to syllable combinations}
% review the mechanism of the tensor framework and its operational implications. In particular the substraction of components.
