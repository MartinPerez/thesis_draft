\begin{fullwidth}
\chapter{\label{ch:super_intro}
The superposition principle and the neuroimaging of language}
\end{fullwidth}

\begin{chabstract}

In this chapter we introduce the problem of variable binding from a connectionist perspective. We show how this problem can be addressed by Smolensky's tensor framework and introduce its superposition principle. Finally we review the neuroimaging literature indirectly linked to the superposition principle and directly linked to language and morphology.

\end{chabstract}


\section{The problem of variable binding from a connectionist perspective}
% review the problem of binding from a connectionist perspective


\section{Smolensky's tensor framework and the superposition principle}
% review the mechanism of the tensor framework and its operational implications. In particular the substraction of components.


\section{Neuroimaging of superposition}
% Superposition related experiments, not necessarily language related?
% The bag of words classification that works well in practice? Counterevidence of superposition
% trivial evidence in sensory systems like retinotopy?

\section{Neuroimaging of language (with emphasis on morphology)}
% Quick introduction to the language network with emphasis of the Federonko language network localizers metanalysis.
% Since we are working with syllables, the morphology effects metaanalysis literature is relevant. Coordinates of previous effects. Would we expect superposition in some type of effects? (Careful not to overlap with discussion)
